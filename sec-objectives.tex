\section{Objetivos}\label{chap:obj}

\subsection*{Objetivo General}\label{sec:obj-g}
El objetivo de esta memoria es extender JUnit-Quickcheck para especificar
propiedades de eficiencia. El usuario debería poder especificar el tipo de
recurso a medir y Quickcheck deberá generar entradas aleatorias crecientes y
los espías necesarios para medir los recursos y ejecutar el programa con ellos
para que finalmente se valide si la propiedad se cumple con las medidas tomadas
en cada ejecución.

\subsubsection*{Ejemplo de uso}
Un ejemplo de uso de esta herramienta puede ser el siguiente, supongamos que un
usuario quiere testear su implementación de Mergesort. Sabemos que en el peor
caso, Mergesort hace $\mathcal{O}(n\log{n})$ comparaciones donde $n$ es el número
de objetos a ordenar.
Por lo tanto al describir la propiedad con el decorador de Quickcheck
\texttt{@Property}, el usuario utiliza las clases de utilidad
\texttt{ComparableMarker} y \texttt{DislProfiler} para instrumentar la ejecución
del test y contar el número de comparaciones que se efectúan.

La siguiente clase es un ejemplo de una definición de utilidad para contar el
número de comparaciones cada vez que se ejecuta el método \textit{compare} en una
ejecución.

\begin{verbatim}

static class ComparisonsProfiler extends DislProfiler {
  @SyntheticLocal
  private long comparisonCount;

  @After(marker=ComparableMarker.class)
  static void onCompare() {
    comparisonCount += 1;
  }

  public ComparisonsProfiler() {
    comparisonCount = 0;
  }

  public long getComparisonCount() {
    return comparisonCount;
  }
}
\end{verbatim}

El siguiente es un ejemplo del test, definido de la misma forma que los tests
de JUnit-Quickcheck pero usando las nuevas utilidades.
La clase \textit{Profiling} se encarga de inyectar en el bytecode los espías
necesarios creados por la clase \texttt{ComparableMarker}.

\begin{verbatim}

@RunWith(JUnitQuickcheck.class)
public static class MergeSortComparisons<T extends Comparable> {

  static ComparisonProfiler profiler;

  @Before
  public void initProfiler() {
    profiler = new ComparisonProfiler();
  }

  @Property
  public void maxComparisons(List<T> objects) {
    Profiling.inject(profiler);
    int n = objects.length();
    mergeSort(objects);
    assertThat(
      profiler.comparisonCount,
      is(lessOrEqual(n * Math.log(n)))
    );
  }
}
\end{verbatim}
\newpage

\linespread{1.5}
\subsection*{Objetivos Específicos}\label{sec:obj-e}
\begin{itemize}
\item Crear una interfaz para ejecutar las funciones de DiSL desde Quickcheck.
\item Crear clases de utilidad para medir:
\subitem Tiempo de ejecución.
\subitem Número de comparaciones.
\subitem Métodos llamados.
\subitem Memoria usada.
\item Transparentar el backend para que el usuario pueda definir sus propias
  clases de utilidad.
\item Extender Quickcheck para hacer uso de las clases de utilidad.
\item Realizar un estudio sobre benchmark para validar que la medición de
  métricas es correcta.
\item Publicar el código fuente en un repositorio público.
\item Publicar la librería en un repositorio de paquetes público.
\item Crear README y documentación.
\end{itemize}

\subsection*{Objetivos Específicos}\label{sec:obj-e}
\begin{itemize}
\item Crear una interfaz para ejecutar las funciones de DiSL desde Quickcheck.
\item Crear clases de utilidad para medir:
\subitem Tiempo de ejecución.
\subitem Número de comparaciones.
\subitem Métodos llamados.
\subitem Memoria usada.
\item Transparentar el backend para que el usuario pueda definir sus propias
  clases de utilidad.
\item Extender Quickcheck para hacer uso de las clases de utilidad.
\item Realizar un estudio sobre benchmark para validar que la medición de
  métricas es correcta.
\item Publicar el código fuente en un repositorio público.
\item Publicar la librería en un repositorio de paquetes público.
\item Crear README y documentación.
\end{itemize}
