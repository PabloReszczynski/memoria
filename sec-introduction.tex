\section{Introducción}
Entender y poder caracterizar las propiedades de eficiencia de un
sistema de software es una herramienta fundamental para evaluar sus cualidades y atributos.
En particular, el análisis del tiempo de ejecución es de vital importancia en
esta labor, así como otros recursos como cantidad de memoria usada o accesos a
las entradas y salidas del sistema. El objetivo de esta memoria es extender las
técnicas de testing actuales para poder validar propiedades sobre el uso de
recursos de programas.

Para eso, se extenderá JUnit-Quickcheck~\cite{pholser}, una librería
de testing aleatorio para Java usando DiSL, una librería de instrumentación para
realizar analisis de la ejecución de programas.
Al finalizar el trabajo de memoria se espera haber construído
un prototipo completamente funcional para testear aleatoriamente propiedades
varias sobre uso de recursos de programas y que haya demostrado la utilidad del
mismo a través de un conjunto de estudio de casos.
