\section{Estado del Arte}\label{chap:eoa}

\subsection*{Quickcheck}
\textbf{Quickcheck}~\cite{Claessen:2011:QLT:1988042.1988046} es una herramienta para
testear propiedades generales de programas.
Inicialmente creado para el lenguaje de programación Haskell, pero hoy en día
ha sido portada a la mayoría de los lenguajes populares.
Los usuarios de Quickcheck describen propiedades de sus programas usando
un DSL\footnote{Domain Specific Language}. Quickcheck luego prueba que esas
propiedades se mantienen ciertas generando una gran cantidad de entradas
aleatorias para el programa en cuestión.
Para casos más complejos, los usuarios tienen la opción de crear generadores de
datos más específicos.

Para el desarrollo de este trabajo se extenderá JUnit-Quickcheck, una
implementación de Quickcheck de código abierto para Java basado en el
framework de testing JUnit.

\subsection*{DiSL}
\textbf{DiSL}~\cite{Marek:2012:DDL:2162049.2162077} es una herramienta y DSL de
instrumentación\footnote{Instrumentación de software significa la habilidad de monitorear o medir la
ejecución de un programa mientras este corre.} para la máquina virtual de Java.
DiSL funciona inyectando código en el bytecode de los programas de Java para obtener análisis
más profundos sobre la ejecución de los programas como perfilar (profiling),
depuración e ingeniería inversa, esto sin modificar el resultado final de la
ejecución que se está analizando.

Para el desarrollo de este trabajo, nos interesa usar DiSL para especificar los
tipos de recursos que un usuario quiera modelar. Un ejemplo es tiempo de
ejecución, cantidad de mensajes que un objeto envía o la cantidad de accesos a
memoria secundaria, entre otros.
El código fuente es abierto y está disponible por lo que la herramienta se
puede extender de ser necesario.

DiSL es superior a otras librerías de perfilamiento porque permite instrumentar
cualquier punto del bytecode, y el código generado es eficiente y no hace
operaciones costosas o que puedan modificar el funcionamiento de los
métodos~\cite{10.1007/978-3-642-35182-2_18}.
