\section{Solución Propuesta}\label{chap:sol}
% Que quieres lograr? %
Lo que se quiere lograr es tener una librería para Java para realizar tests de
aleatorios para demostrar propiedades de eficiencia en programas.
Actualmente existen las herramientas para crear tal librería -DiSL y Quickcheck
en este caso- pero es un trabajo que aún no se ha hecho y que tampoco es trivial.
Esto es porque es necesario abstraer la funcionalidad de DiSL para poder ser
usada por Quickcheck.

Con la librería el usuario podrá especificar un test de eficiencia con una
métrica y una heurística de cuán eficiente tiene que ser su programa. Quickcheck
correrá el test con cientos de inputs generados aleatoriamente y en cada
ejecución revisará con DiSL que la propiedad general del test se mantiene.
Si en alguna ejecución la propiedad no se cumple, el test fallará. En caso
contrario el test pasará.

La entrega final del trabajo consistirá en una librería documentada para Java
que permita a los usuarios testear propiedades de eficiencia de sus programas
usando distintas métricas y con entradas generadas de forma aleatoria.
La librería vendrá con métricas por defecto como tiempo de ejecución y conteo
de comparaciones, además de funcionalidad para que los usuarios creen sus
propias métricas.

La librería estará disponible en un repositorio de paquetes de Maven para
poder ser integrado de manera sencilla por los usuarios en sus proyectos de
Java.

